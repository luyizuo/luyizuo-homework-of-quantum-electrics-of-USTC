\documentclass{ctexart}

% 基本信息
\title{\vspace{-3cm}mytitle}
\author{LiLi}
\date{\today}

% 页面
\usepackage{geometry} % 设置页边距
\geometry{a4paper,centering,scale=0.8}

% 文字
\usepackage{lmodern} % 消除字体问题,;有时管用

% 颜色
% \usepackage{color} % 设置字体颜色
\usepackage[dvipsnames]{xcolor}

% 公式
% \usepackage{boondox-calo} % 产生小写花体字母 但是大写的花体cal 被改的不好看了
\usepackage{amsmath,mathrsfs,amsfonts,amssymb} % 分别是  ,花体字  ,  。
\usepackage{commath} % 定义微分算子 d 
\usepackage{bm} % \bm 数学符号的斜体


\usepackage{amsthm} % 定理和证明
\newtheorem{theorem}{Theorem}[section]
% \newtheorem{lemma}[theorem]{Lemma}
% \newtheorem{proposition}[theorem]{Proposition}
\newtheorem{corollary}[theorem]{Corollary}
\newtheorem{definition}{Definition}[section]
\newtheorem{remark}{Remark}[section]
% \newtheorem{example}{Example}[section] % 还是没有我想要的框框效果
% \newenvironment{solution}{\begin{proof}[Solution]}{\end{proof}}

\usepackage[most]{tcolorbox} % 加most 可以解决编译时遇到的 I do not know the key '/tcb/enhanced' 之类的问题


% 图片
\usepackage{graphicx} % 和graphics 命令都一样 可选变量不一样
\usepackage{float}
% \usepackage[format=hang]{caption} % 看个人需要 textfont=it,font=small

\usepackage{tikz}
\usetikzlibrary{3d,perspective}
\usetikzlibrary{angles,quotes}


% 自定义的环境
\renewcommand\emph[1]{\colorbox{YellowGreen}{#1}}



% 自定义的环境
\newenvironment{objectives}
{\noindent\rule{\textwidth}{2pt} % begin part
\textbf{Learning objectives}
\vspace{-5pt} \\ \rule{\textwidth}{1pt}}
{ % end part % 中间的空行是为了适应环境中的内容结尾可能有分段(比如itemize)也可能没分段(比如单纯的文字)
    
\noindent\rule{\textwidth}{1pt}} % 不要动中间的空行,可能弄坏了


\newenvironment{example}[1][] % 这个设置取自文档p166
    {\begin{tcolorbox}
    [enhanced,
    title=Example: #1, % #1 前面的空格加多少个都视为一个,不错
    colframe=red!50!black,
    colback=red!3!white,%  % 整体内容区域的颜色 加了微量的红色更加跟其他内容区分开来
    arc=1mm,
    colbacktitle=red!10!white,
    fonttitle=\bfseries,
    coltitle=red!50!black,
    attach boxed title to top left={xshift=3.2mm,yshift=-0.50mm},
    boxed title style={skin=enhancedfirst jigsaw,
        size=small,arc=1mm,bottom=-1mm,
        interior style={fill=none,% example 这个字样的背景色
            top color=red!30!white, % 从上到下还是个渐变色
            bottom color=red!20!white}},
    breakable]} % breakable 可以使框框分页,但是变成了两个框,再加上前面的enhanced 就可以既分页,又还是一个框
{\end{tcolorbox}}





% \usepackage{lipsum} % 生成一段文字
\usepackage{pdfpages} % 插入一个pdf的某些页面 usage:\includepdf[pages={1-3}]{1.pdf}




% self-define
\newcommand{\etal}{\textit{et al}., }
\newcommand{\ie}{\textit{i}.\textit{e}., }
\newcommand{\eg}{\textit{e}.\textit{g}.\ }
\newcommand{\bbinom}[2]{\genfrac{[}{]}{0pt}{}{#1}{#2}} % '['还可以是多个字符的组合 不小心试出来的
\newcommand{\啊}{啊啊啊啊啊啊啊啊啊啊啊啊啊啊啊啊啊啊啊啊啊啊啊啊啊啊啊啊啊啊啊啊啊啊啊啊啊啊啊啊啊啊啊啊啊啊啊啊啊啊啊啊啊啊啊啊啊啊啊啊啊啊啊啊啊啊啊啊啊啊啊啊啊啊啊啊啊啊啊啊啊啊啊啊啊啊啊啊啊啊啊啊啊啊啊啊啊啊啊啊啊啊啊啊啊啊啊啊啊啊啊啊啊啊啊啊啊啊啊啊啊啊啊啊啊啊啊啊啊啊啊啊啊啊啊啊啊啊啊啊啊啊啊啊啊啊啊啊啊啊啊啊啊啊啊啊啊啊啊啊啊啊啊啊啊啊啊啊啊啊啊啊啊啊啊啊啊啊啊啊啊啊啊啊啊啊啊啊啊啊啊啊啊啊啊啊啊啊啊啊啊啊啊啊啊啊啊啊啊啊啊啊啊啊啊啊啊啊啊啊}

% 更细的修补
\allowdisplaybreaks % break the equations in 'align' environment through pages
\everymath{\vadjust{\nobreak\null}} % Allow line breaks but not page breaks in inline formulas 这个不知道工作原理是什么

% \usepackage{breqn} 专门解决break equation 的宏包



\begin{document}
    
\section*{部分习题}

% 6.7

% $$
% T=
% \begin{pmatrix}
%     1&d_2\\    
%     0&1\\    
% \end{pmatrix}
% \begin{pmatrix}
% 1&0\\    
% -1/f&1\\    
% \end{pmatrix}
% \begin{pmatrix}
% 1&d_1\\    
% 0&1\\    
% \end{pmatrix}
% $$



6.11

考虑波包

\[A(z, t)=\int_{-\infty}^{\infty} G(\Delta \omega) e^{i[\Delta \omega t-\Delta k(\omega) z]} d(\Delta\omega) \]
作二阶展开

\[\Delta k(\omega)=\left.\frac{d k}{d \omega}\right|_{\omega=\omega_{0}} \cdot \Delta \omega+\frac{1}{2}\left.\frac{d^{2} k}{d \omega^{2}}\right|_{\omega=\omega_{0}} \cdot(\Delta \omega)^{2} \]
令\(a=\frac{1}{2} \frac{d^{2} k}{d \omega^{2}} \Big|_{\omega=\omega_0}\)则有

\[A(z, t)=\int_{-\infty}^{\infty} G(\Delta \omega) e^{i\left[\Delta \omega t-\frac{z}{v_{g}} \Delta \omega+a(\Delta \omega)^{2} z\right]} \]
取高斯波包\(G(\Delta \omega)=\sqrt{\frac{\pi}{2}} e^{-\Delta \omega^{2} / 4 \alpha}\), 其中脉冲宽度\(\tau_{0}=\sqrt{\frac{2 \ln 2}{\alpha}}\), 带入积分计算得到

\[A(z, t)=N \exp [i \varphi(z, t)] \exp \left(\frac{\left(t-\frac{z}{v g}\right)^{2}}{\frac{1}{\alpha}+16 \alpha a^{2} z^{2}}\right) \]
其中\(N\)为归一化因子, 而\(\varphi(z,t)\)是一实函数, 由此得脉冲宽度

\[\tau(z)=\sqrt{\frac{2 \ln 2}{\alpha}} \sqrt{1+16 a^{2} \alpha^{2} z^{2}}\simeq \tau_{0} 4 |a|\alpha z=4 \ln 2 \frac{L}{\tau_{0}} \frac{d^{2} k}{d w^{2}}\bigg|_{w=w_{0}} \]
最后的\(z\)取\(L\), 而且

\[\frac{d^{2} k}{d \omega^{2}}\bigg|_{\omega=\omega_{0}}= \frac{d}{d \omega} \frac{d k}{d \omega}=-\frac{1}{v_{g}^{2}} \frac{d v_{g}}{d \omega} \]
或者, 频率为\(\omega_{0}+\frac{1}{2} \Delta \omega\)的部分和\(\omega_{0}-\frac{1}{2} \Delta \omega\)的部分, 在不同时刻到达\(L\)处, 时间差为

\[\Delta\tau=\frac{L}{v_{g}\left(\omega+\frac{1}{2} \Delta w\right)}-\frac{L}{v_{g}\left(\omega-\frac{1}{2} \Delta \omega\right)}=\left(\frac{1}{v_g+\frac{d v_{g}}{d \omega} \big| _{\omega=\omega_{0}} \cdot \frac{1}{2} \Delta \omega}-\frac{1}{v_g-\frac{d v_{g}}{d \omega} \big| _{\omega=\omega_{0}} \cdot \frac{1}{2} \Delta \omega}\right)\\=-\frac{L}{\nu_{g}^{2}}\left.\frac{d v_{g}}{d \omega}\right|_{\omega=\omega_{0}} \cdot \Delta \omega \]
其中\(\Delta \omega=\frac{1}{\pi \tau_{0}}\), 于是得到题中给出的展宽.

6.12

光斑尺寸由下式决定,

\[\omega=\sqrt{\frac{\lambda}{\pi}}\left(\frac{1}{n n_{2}}\right)^{\frac{1}{4}} \]
而展宽后的总脉冲宽度为

\[\tau+\Delta \tau=\frac{L}{v_{g}^{2}} \frac{n_{2} / n}{k^{3}}(l+m+1)^{2} \frac{1}{\pi \tau}+\tau \]
从而每秒最多发送的脉冲数量, 由上面的脉冲宽度决定, 即\(1/(\tau+\Delta\tau)\).

6.13

令\(\psi(x, y)=f(x) g(y)\), 则分离变量可得

\[\left\{\begin{array}{l}{\frac{f^{\prime \prime}(x)}{f(x)}-k^{2} \frac{n_{2}}{n} x^{2}=-\lambda_x} \\ {\frac{g^{\prime \prime}(y)}{g(y)}-k^{2} \frac{n_{2}}{n} y^{2}=-\lambda_{y}} \\ {k^{2}-\beta^{2}-\lambda_{x}-\lambda_{y}=0}\end{array}\right. \]
再引入下面变量进行无量纲化处理

\[\xi=\frac{\sqrt{2} x}{\omega}, \omega=\sqrt{\frac{\lambda}{\pi}}\left(\frac{1}{n \pi_{2}}\right)^{\frac{1}{4}} \]
得到\(f^{\prime \prime}(\xi)+\left(\lambda^{\prime}-\xi^{2}\right) f(\xi)=0\), 所以\(\lambda^{\prime}=2 l+1\), 而

\[f(x)=H\left(\frac{\sqrt{2} x}{\omega}\right) e^{-\left(\frac{x}{\omega}\right)^{2}} \]
对\(y\)也类似处理, 有

\[\psi(x, y)=E_{(l, m)} H_{l}\left(\sqrt{2} \frac{x}{\omega}\right) H_{m}\left(\sqrt{2} \frac{y}{\omega}\right) e^{-\frac{x^{2}+y^{2}}{\omega^{2}}} \]
再利用\(k^{2}-\beta^{2}-\lambda_{x}-\lambda_{y}=0\)可得

\[\beta_{l, m}=k \sqrt{1-\frac{2}{k} \sqrt{\frac{n_{2}}{n}}(l+m+1)} \]



7.1
由PPT知
\begin{gather}
    Q=\omega t_{c}\\
    t_{c}=\frac{n l}{c\left[\alpha l+\left(1-\sqrt{R_{1} R_{2}}\right)\right]}
\end{gather}
假设对光学腔和微波腔均有
$$
\alpha=0, n=1, R_1=R_2=R
$$
则
$$
t_c=\frac{l}{c(1-R)}
$$
所以差异仅仅在$\omega$上. 因此光学腔的\(Q\)值比微波腔高\(3\)到\(6\)个数量级.

7.2
Design a resonator with $R_1=-20$ cm, $R_2=32$ cm, $l=16$ cm, $\lambda=10^{-4}$ cm.
Determine
\begin{enumerate}
    \item the minimum spot size$\omega_0$
    \item Its location
    \item The spot size $\omega_1$ and $\omega_2$ at the mirrors
    \item the ratios of $\omega_0$, $\omega_1$, and $\omega_2$ to their respective confocal $(-R_1 = R_2 = l)$ values.
\end{enumerate}

1.由PPT知
$$
\left\{
\begin{aligned}
    &z_0^2=\frac{l(R_2-R_1-l)(l+R_1)(l-R_2)}{(2l+R_1-R_2)^2}\\
    &z_0=\frac{\pi \omega_0^2}{\lambda}
\end{aligned}
\right.
$$
将数值代入公式得
\begin{align*}
    z_0^2
    &=\frac{l(R_2-R_1-l)(l+R_1)(l-R_2)}{(2l+R_1-R_2)^2}\\
    &=\frac{16\cdot36(-4)(-16)}{(-20)^2}\\
    &=\frac{2304}{25}\\
    z_0
    &=9.6 \ \mathrm{cm}\\
    \omega_0
    &=\sqrt{\frac{\lambda z_0}{\pi}}\\
    &=\sqrt{\frac{10^{-4} \cdot 9.6}{\pi}}\\
    &=1.75\times10^{-2} \ \mathrm{cm}
\end{align*}

2.由PPT知
$$
\left\{
\begin{aligned}
    &z_1=\frac{1}{2}\left(R_1\pm \sqrt{R_1^2-4z_0^2}\right)\\
    &z_2=\frac{1}{2}\left(R_2\pm \sqrt{R_2^2-4z_0^2}\right)
\end{aligned}
\right.
$$
Note that $z_1, z_2$ will have two value from which we choose the value exactly satisfying $z_2-z_1=l$ and that is the expected result.
% where in this case\dots

代入数值得
\begin{align*}
    z_1
    &=\frac{1}{2}\left(R_1\pm \sqrt{R_1^2-4z_0^2}\right)
    =\frac{1}{2}\left(-20\pm \sqrt{20^2-4\cdot 9.6^2}\right)
    =\frac{1}{2}\left(-20\pm 5.6\right)
    =-7.2 \ \mathrm{or}\  -12.8 \ \mathrm{cm}\\
    z_2
    &=\frac{1}{2}\left(32\pm \sqrt{32^2-4\cdot 9.6^2}\right)
    =\frac{1}{2}\left(32\pm 25.6\right)
    =28.8 \ \mathrm{or}\ 3.2 \ \mathrm{cm}
\end{align*}
so 
$$
z_1=-12.8 \ \mathrm{cm}, \quad z_2=3.2 \ \mathrm{cm}
$$



% 中文版
% \begin{enumerate}
%     \item 最小光斑尺寸$\omega_0$;
%     \item 最小光斑的位置;
%     \item 镜面光斑尺寸$\omega_1$和$\omega_2$;
%     \item 和w2分别与共焦腔(B=一五2=)相应值的比。
% \end{enumerate}

3.use the Gaussian parameter relation
$$
\omega(z)=\omega_0 \left(1+z^2/z_0^2\right)^{1/2}
$$
代入数值得
\begin{align*}
   &\omega_1=\omega_0 \left(1+z_1^2/z_0^2\right)^{1/2}
   =\omega_0(1+12.8^2/9.6^2)^{1/2}=\frac{5}{3}\omega_0
   \ (\ =2.92\times10^{-2} \ \mathrm{cm})\\
   &\omega_2=\omega_0 \left(1+z_2^2/z_0^2\right)^{1/2}
   =\omega_0(1+3.2^2/9.6^2)^{1/2}=\frac{\sqrt{10}}{3}\omega_0
   \ (\ =1.84\times10^{-2} \ \mathrm{cm})
\end{align*}

4.由PPT知
$$
\left\{
\begin{aligned}
    &\omega_{0,\mathrm{conf}}=\sqrt{l/k}\\
    &\omega_{1,2,\mathrm{conf}}=\sqrt{2}\omega_{0,\mathrm{conf}}
\end{aligned}
\right.
$$

代入数值得
\begin{align*}
    &\omega_{0,\mathrm{conf}}=\sqrt{\frac{\lambda l}{2\pi}}=\sqrt{\frac{10^{-4}\cdot 16}{2\pi}}
    =1.60\times10^{-2} \ \mathrm{cm}\\
    &\omega_{1,2,\mathrm{conf}}=\sqrt{2}\omega_{0,\mathrm{conf}}
    =2.26\times10^{-2} \ \mathrm{cm}
\end{align*}



% 7.4

% 令\(\frac{l}{R}=x, \frac{l}{R_{2}}=y\), 则稳定条件的方程约束写为\(0 \leq(1-x)(1-y) \leq 1\), 
% 注意\(\frac{1}{x}+\frac{1}{y}=1\)和\((x-1)(y-1)=1\)是同一条曲线. 
% 边界\((x-1)(y-1)=1\)把平面分为6块, 对每一块要么全部满足约束要么全部违反约束.


7.5

见图,计算 $1\to2$ 这一个周期的传输矩阵$T$

在二元周期透镜中令 $f_2=f, f_1=-f$

得
$$
T=
\begin{pmatrix}1-l/f&2l-l^2/f\\-l/f^2&l/f-l^2/f^2+1\end{pmatrix}
$$
$$
b=\frac{1}{2}(A+D)=2-l^2/f^2
$$


% 或者
% \begin{align*}
% T&=
% \begin{pmatrix}1&0\\1/f&1\end{pmatrix}
% \begin{pmatrix}1&l\\0&1\end{pmatrix}
% \begin{pmatrix}1&0\\-1/f&1\end{pmatrix}
% \begin{pmatrix}1&l\\0&1\end{pmatrix}\\
% &=\begin{pmatrix}1&0\\1/f&1\end{pmatrix}
% \begin{pmatrix}1&l\\0&1\end{pmatrix}
% \begin{pmatrix}1&l\\-1/f&-l/f+1\end{pmatrix}\\ 
% &=\begin{pmatrix}1&0\\1/f&1\end{pmatrix}   
% \begin{pmatrix}1-l/f&2l-l^2/f\\-1/f&-l/f+1\end{pmatrix}\\
% &=\begin{pmatrix}1-l/f&2l-l^2/f\\-l/f^2&l/f-l^2/f^2+1\end{pmatrix}
% \end{align*}
% so 
% $$
% r_{n+1}=(1-\frac{l}{f})r_n-\frac{l}{f^2}r'_n
% $$



最后是正弦震荡,题目说的“净聚焦”有误导性


7.6
对于对称腔,束腰位于腔中心,因此有
$$
z_2=-z_1=\frac{l}{2}, \quad R_2=-R_1=R
$$

代入
$$
z_0^2=\frac{l(R_2-R_1-l)(l+R_1)(l-R_2)}{(2l+R_1-R_2)^2}
$$
得
$$
z_0^2=\frac{1}{4}l(2R-l)
$$
再由高斯光束的参数关系
$$
\omega_0^2=\frac{2z_0}{k}=\frac{2}{k}\sqrt{l(2R-l)}
$$
$$
\omega_{1,2}
=\omega_0\left(1+z_{1,2}^2/z_0^2\right)^{1/2}
=\omega_0\left(1+\frac{l}{2R-l}\right)^{1/2}
=\omega_0\left(\frac{2R}{2R-l}\right)^{1/2}
$$


7.7

由


$$
\left\{\begin{array}{l}{R_{1}=z_{1}+\frac{z_{0}^{2}}{z_{1}}} \\ {R_{2}=z_{2}+\frac{z_{0}^{2}}{z_{2}}}\end{array}\right. 
$$

代入
$$
g_{1} g_{2}=\left(1-\frac{l}{R_{1}}\right)\left(1-\frac{l}{R_{2}}\right)
$$
时注意$R_1$要变号,得到

\[g_{1} g_{2}=\frac{\left(z_{0}^{2}+z_{1} z_{2}\right)^{2}}{\left(z_{1}^{2}+z_{0}^{2}\right)\left(z_{2}^{2}+z_{0}^{2}\right)} \geqslant 0 \]

再设\(g_{1} g_{2} \leqslant 1\), 并展开消项, 得到
\(2 z_{1} z_{2} \leqslant z_{1}^{2}+z_{2}^{2}\), 
所以确实有\(g_{1} g_{2} \leqslant 1\).

7.9

以第二面反射镜处为参考面, 则传输矩阵为

\[
\left(\begin{array}{ll}{1} & {l} \\ {0} & {1}\end{array}\right)
\left(\begin{array}{cc}{1} & {0} \\ {-\frac{2}{R_{1}}} & {1}\end{array}\right)
\left(\begin{array}{ll}{1} & {l} \\ {0} & {1}\end{array}\right)
\left(\begin{array}{cc}{1} & {0} \\ {-\frac{2}{R_{2}}} & {1}\end{array}\right)
=\left(\begin{array}{ll}{\frac{4 l^{2}+R_{1} R_{2}-2 l\left(2 R_{1}+R_{2}\right)}{R_{1} R_{2}}} 
& {l\left(2-\frac{2 l}{R l}\right)} \\ 
{-2 \frac{\left(-2 l+R_{1}+R_{2}\right)}{R_{1} R_{2}}} 
& {l-\frac{2 l}{R l}}\end{array}\right) 
\]
由7.2-7式可得
$$
R(z=\text{第二面反射镜处})=\frac{2B}{D-A}
$$
将矩阵元带入化简得\(\frac{2B}{D-A}=R_2\). 

对于\(R_1\)镜面, 选取参考面在\(R_1\)处, 也可以得到类似证明.

7.10

由7.2-3可得
\[\frac{1}{q}=\frac{C+D /q}{A+B/q} \]
设函数
\[
f(x)=\frac{C+D x}{A+B x}, \delta f(x)=f(x+\Delta x)-f(x) \simeq \frac{\Delta x}{(A+B x)^{2}} 
\]
在上式右边的\(x\)中带入稳态解7.2-5得
\[
\delta f=e^{\mp i2 \theta}\Delta x 
\]
其中\(\Delta x\)代表\(\Delta(1/q)\), 而\(\delta f\)代表\(\delta(1/q)\). 显然当满足稳定条件的时候, 即\(\theta\)是实数, 则有
\[\left|\delta\left(\frac{1}{q}\right)\right|=\left|\Delta\left(\frac{1}{q}\right)\right| \]












8.3

按照\(r=r_0\cos(\omega t)\)简谐振动的电子, 能量为

\[E=\frac{1}{2} m \dot{r}^{2}+\frac{1}{2} m \omega^{2} r^{2}=\frac{1}{2} m \omega^{2} r_{0}^{2} \]
按照拉莫尔公式, 辐射功率为

\[P=\frac{e^{2} \dot{r}^{2}}{6 \pi \varepsilon_{0} c^{3}} \]
于是寿命为

\[t_{\text{经典}}=\frac{E}{P}=\frac{6 \pi \varepsilon_{0} c^{3} m}{e^{2} \omega^{2}}=\frac{3 \varepsilon_{0} c^{3} m}{2 \pi e^{2} \nu^{2}} \]


8.5

令\(\dot{\sigma}_{21}=0, \dot{\rho}_{11}-\dot{\rho}_{22}=0\)得到关于$\sigma_{21}$和$\rho_{11}-\rho_{22}$达到稳态时的方程组
\begin{gather*}
\mathrm{i}\left(\omega-\omega_{0}\right) \sigma_{21}+\mathrm{i} \Omega\left(\rho_{11}-\rho_{22}\right)-\frac{\sigma_{21}}{T_{2}}=0
\tag*{\textcircled{1}}\label{1}\\ 
2\mathrm{i}  \Omega\left(\sigma_{21}-\sigma_{21}^{*}\right)-\frac{\rho_{11}-\rho_{22}-\left(\rho_{11}-\rho_{22}\right)_{0}}{\tau}=0 
\tag*{\textcircled{2}}\label{2}   
\end{gather*}

\begin{theorem}
上述方程组的解为
\begin{gather*}
\text{Im}\left[\sigma_{21}\right]=\frac{\Omega T_{2}\left(\rho_{11}-\rho_{22}\right)_0}{1+\left(\omega-\omega_{0}\right)^{2} T_{2}^{2}+4 \Omega^{2} T_{2} \tau}\\ 
\text{Re}\left[\sigma_{21}\right]=\frac{-\left(\omega-\omega_{0}\right) \Omega T_{2}^{2}\left(\rho_{11}-\rho_{22}\right)_{0}}{1+\left(\omega-\omega_{0}\right)^{2} T_{2}^{2}+4 \Omega^{2} T_{2} \tau}\\
\rho_{11}-\rho_{22}=\left(\rho_{11}-\rho_{22}\right)_0 \frac{1+\left(\omega-\omega_{0}\right)^{2} T_{2}^2}{4 \Omega^{2} T_{2} \tau+1+\left(\omega-\omega_{0}\right)^{2} T_{2}^{2}}
\end{gather*}    
\end{theorem}
\begin{proof}

由\ref{1}解出$\sigma_{21}$

\begin{equation}
\sigma_{21}=\frac{\mathrm{i} \Omega\left(\rho_{11}-\rho_{22}\right)}{\frac{1}{T_{2}}-\mathrm{i}\left(\omega-\omega_{0}\right)}
\tag*{\textcircled{3}}\label{3} 
\end{equation}

把\ref{3}代回\ref{2}解得
\begin{equation}
\rho_{11}-\rho_{22}=\left(\rho_{11}-\rho_{22}\right)_0 \frac{1+\left(\omega-\omega_{0}\right)^{2} T_{2}^2}{4 \Omega^{2} T_{2} \tau+1+\left(\omega-\omega_{0}\right)^{2} T_{2}^{2}} 
\tag*{\textcircled{4}}\label{4}    
\end{equation}
把\ref{4}代回\ref{1}解得
\begin{gather*}
\text{Im}\left[\sigma_{21}\right]=\frac{\Omega T_{2}\left(\rho_{11}-\rho_{22}\right)_0}{1+\left(\omega-\omega_{0}\right)^{2} T_{2}^{2}+4 \Omega^{2} T_{2} \tau}\\ 
\text{Re}\left[\sigma_{21}\right]=\frac{-\left(\omega-\omega_{0}\right) \Omega T_{2}^{2}\left(\rho_{11}-\rho_{22}\right)_{0}}{1+\left(\omega-\omega_{0}\right)^{2} T_{2}^{2}+4 \Omega^{2} T_{2} \tau}
\end{gather*}
\end{proof}



8.6

\[\chi(\omega)=\chi^{\prime}(\omega)-i \chi^{\prime \prime}(\omega)\\=\frac{\mu \Delta N_{0}}{\varepsilon_{0} \hbar}\frac{\omega-\left[\omega_{0}-\left(i / T_{2}\right)\right)}{\left\{\omega-\left[\omega_{0}-\left(i / T_{2}\right) \sqrt{\left(1+s^{2}\right)}\right]\right\}\left\{\omega-\left[\omega_{0}+\left(i / T_{2}\right) \sqrt{1+s^{2}}\right]\right\}} \]
在\(s^{2}=\mu^{2} E_0^{2} T_{2} \tau / \hbar^{2} \propto E_{0}^{4}=0\)时,

\[\chi(\omega)=-\frac{\mu \Delta N_{0}}{\varepsilon_{0} \hbar} \frac{1}{\omega-\left[\omega_{0}+\left(i / T_{2}\right)\right]} \]
在\(\omega\)下半平面有单极点, 满足Kramers-Kronig定理的条件.

% 9.1

% 根据

% \[\frac{h \nu V_{m} W_{i}}{K}=n_{m} \]
% \[K=\frac{h \nu V_{m} W_{i}}{n_{m}} \]
% 而\(n_{m}=\frac{I n V_{m}}{c h \nu_{m}}\), 其中\(I\)为光强, 满足8.3-9式,

% \[I=\frac{1}{\lambda^{2} g\left(\nu_{0}\right)} 8 \pi h \nu n^{2} t_{自发}W_i \]
% 带入得到

% \[n_{m}=\frac{n V_{m} W_{i} 8 \pi n^{2} t_{自发} \nu^{2}}{c^{3} g\left(\nu_{0}\right)} \]
% 所以得到

% \[K=\frac{c^{3} h \nu V_{m} W_{i}}{n V_m W_i 8 \pi n^{2} t\nu^2\Delta \nu}
% =\frac{h \nu c^{3}}{8 \pi n^{3} \nu^{2} \Delta \nu t} \]


9.2

由9.1-21式, 当有横向约束时, 内部损耗\(\alpha\)增大, 由

\[\nu=\nu_{m}-\left(\nu-\nu_{0}\right)\frac{c\left[\alpha-1/l \ln\left(r_{1} r_{2}\right)\right]}{2 \pi n \Delta \nu} \]
可知, 激光共振频率将会稍稍远离无源腔共振频率.

利用9.1-18式,

\[\nu_{m}=\frac{m c}{2 n l}+\frac{c}{2 \pi n l}\left(\tan ^{-1} \frac{z_{2}}{z_{0}}-\tan ^{-1} \frac{z_{1}}{z_{0}}\right) \]
再利用\(\arctan a-\arctan b=\arctan \frac{a-b}{1+a b}\)可以化简. 对于共焦腔, 利用6.6-13和7.1-10可得\(z_0=l/2\), 所以有

\[\nu_{m}=\frac{m c}{2 n l}+\frac{c}{2 \pi n l} \arctan \frac{l z_{0}}{z_{0}^{2}+z_{1} z_{2}} \]
利用

\[z_{1}=-\frac{l}{2}, z_{2}=\frac{l}{2} \]
代入可得

\[\arctan \frac{l z_{0}}{z_{0}^{2}+z_{1} z_{2}}=\arctan \frac{\frac{1}{2} l^{2}}{\frac{1}{4} l^{2}-\frac{1}{4} l^{2}}=\frac{\pi}{2} \]
所以共焦腔有

\[\nu_{m}=\frac{m c}{2 n l}+\frac{c}{4 n l} \]
而当\(z_0\gg l\)时, \(\arctan \frac{l z_{0}}{z_{0}^{2}+z_1z_{2}}=0\), 所以,

\[\nu_{m}=\frac{m c}{2 n l} \]
9.3

在\(\text{F-P}\)腔中,即考虑特例\(\omega_0\to\infty\), 则有\(\varphi(z_1),\varphi(z_2)\to0\)

这时
$$
\left\{
\begin{aligned}
    &\nu_{m,\mathrm{dead}}=\frac{c}{2nl}\left[m-\frac{1}{2\pi}(\theta_1+\theta_2)\right]\\
    &\nu_{m}\left[1+\frac{\chi'(\nu_m)}{2n^2}\right]=\nu_{m,\mathrm{dead}}    
\end{aligned}
\right.
$$
则
\begin{align*}
    &\frac{c}{2nl}\left[m-\square\right]=\nu_{m}\left[1+\frac{\chi'(\nu_m)}{2n^2}\right]\\
    &\frac{c}{2nl}\left[m+1-\square\right]=\nu_{m+1}\left[1+\frac{\chi'(\nu_{m+1})}{2n^2}\right]
\end{align*}
作差得到
$$
\frac{c}{2nl}=\nu_{m+1}-\nu_m+\frac{1}{2n^2}\left[\nu_{m+1}\chi'(\nu_{m+1})-\nu_{m}\chi'(\nu_{m})\right]
$$
计算其中的项
\begin{align*}
    \nu_{m+1}\chi'(\nu_{m+1})-\nu_{m}\chi'(\nu_{m})
    &=(\nu_{m+1}-\nu_{m})\chi'(\nu_{m+1})+\nu_{m}\left[\chi'(\nu_{m+1})-\chi'(\nu_{m})\right]\\
    &=(\nu_{m+1}-\nu_{m})\chi'(\nu_{m+1})+\nu_{m}(\nu_{m+1}-\nu_{m})\frac{\dif \chi'(\nu)}{\dif \nu}|_{\nu=\nu_m}\\
    &=(\nu_{m+1}-\nu_{m})\left[\chi'(\nu_{m+1})+\nu_{m}\frac{\dif \chi'(\nu)}{\dif \nu}|_{\nu=\nu_m}\right]
\end{align*}
所以
$$
\frac{c}{2nl}=(\nu_{m+1}-\nu_{m})\left[1+\frac{1}{2n^2}\chi'(\nu_{m+1})+\frac{1}{2n^2}\nu_{m}\frac{\dif \chi'(\nu)}{\dif \nu}|_{\nu=\nu_m}\right]
$$
移项即得答案 (跟答案比多了一项,答案错了吧)

9.4

式9.2-14中, \(\omega_l\)是无源腔的振荡频率, 而\(\omega\)是实际激光频率. 因此实际上\(\omega_l\)对应9.1-17中令\(\chi'=0\)的\(\omega\), 而\(\omega\)则直接就是9.1-17中的\(\omega\). 二者作差得到

\[\omega_{l}-\omega\left[1+\frac{\chi^{\prime}(\nu)}{2 n^{2}}\right]=0 \]
因此

\[\omega_{l}^{2}-\omega^{2} \simeq \omega^{2} \frac{\chi^{\prime}(\nu)}{n^{2}}\tag{$*$} \]
将9.1-14带入9.1-14a中, 得到

\[e^{\left(- k \frac{\chi^{\prime \prime}(\nu)}{n^{2}}-\alpha\right) \cdot l}=\frac{1}{r_{1} r_{2}} \]
即

\[\frac{\omega n}{c} \cdot \frac{x^{\prime \prime}(\nu)}{n^{2}}=-\alpha+\frac{1}{l}\ln(r_1r_2) \]
令

\[\sigma=\left[\alpha-\frac{1}{l} \ln \left(r_{1} r_{2}\right)\right] \frac{c}{n} \cdot \varepsilon \]
则有

\[\frac{\omega\chi^{\prime \prime}(\nu)}{n^{2}}=-\frac{\sigma}{\varepsilon} \]
以\((-i)\)乘以上式并加上\((*)\)式可得

\[\omega_{c}^{2}-\omega^{2}+i \frac{\sigma \omega}{\varepsilon}=\omega^{2} \frac{1}{n^{2}}\left(\chi^{\prime}(\nu)-i \chi^{\prime \prime}(\nu)\right) \]
9.6

稳态时有\(\frac{d}{d t} N_{i}=0\),

\[\left\{\begin{array}{l}{R_{2}-\frac{N_{2}}{t_{1}}-\left(N_{2}-\frac{g_{2}}{g_{1}} N_{1}\right) W_{i}(\nu)=0} \\ {R_{1}-\frac{N_{1}}{t_{1}}+\frac{N_{2}}{t_{21}}+\left(N_{2}-\frac{g_{2}}{g_{1}} N_{1}\right) W_{i}(\nu)=0}\end{array}\right. \]
得到

\[\left\{\begin{array}{l}{N_{1}=\frac{\left(\frac{1}{t_{2}}+W_{i}\right) N_{2}-R_{2}}{\frac{g_{2}}{g_{1}} W_{i}}} \\ {N_{2}=\frac{t_{2} R_{2}+\left(R_{1}+R_{2}\right) t_{1}t_2\frac{g_{2}}{g_{1}} W_{i}}{1+W_{i}\left[t_{2}+(1-\delta) \frac{g_{2}}{g_{1}}\right]}}\end{array}\right. \]
所以

\[\Delta N \equiv N_{2}-\frac{g_{2}}{g_{1}} N_{1}=\frac{R_{2} t_{2}-\left(R_{1}+\delta R_{2}\right) t_{1} \frac{g_{2}}{g_{1}}}{1+\left[t_{2}+(1-\delta) t_{1} \frac{g_{2}}{g_{1}}\right] W_{i}(\nu)} \]


11.1
(a): 
A显示强度调制的基本分量,可看到基频信号。
B显示强度调制的功率谱,可看到一个峰值很高的拍频信号。
C显示光场功率谱,可看到不同谱线对应的强度,中间高,两边低。
D显示光强,可看到多个等强度等间隔的高峰值脉冲。

(b):
锁模可使拍频信号功率增加N倍。



14.2
对于横向电光调制器,双折射会产生附加的相位差$\varphi=(n_o-n_e)k_0l$, 因为温度变化时,$n_o,n_e$会变化,所以附加相位也会随温度发生漂移,影响调制器的正常工作

14.3
利用公式
$$
\sin [a\sin t]=2\sum_{n=1}^{\infty}\mathrm{J}_{2n-1}(a)\sin[(2n-1)t]
$$

则
$$
\frac{I_o}{I_i}=\frac{1}{2}\left[1+\sin\left(\Gamma_m\sin \omega_m t\right)\right]
=\frac{1}{2}+\sum_{n=1}^{\infty}\mathrm{J}_{2n-1}(\Gamma_m)\sin[(2n-1)\omega_mt]
$$

$$
I_3=\mathrm{J}_3(\Gamma_m)\sin(3\omega_mt)  \to \frac{1}{2}\mathrm{J}_3(\Gamma_m)
$$
$$
I_1=\mathrm{J}_1(\Gamma_m)\sin(\omega_mt) \to \frac{1}{2}\mathrm{J}_1(\Gamma_m)
$$
则
$$
\frac{I_3}{I_1}=\frac{\mathrm{J}_3(\Gamma_m)}{\mathrm{J}_1(\Gamma_m)}
$$
要使
$$
f(\Gamma_m)=\frac{\mathrm{J}_3(\Gamma_m)}{\mathrm{J}_1(\Gamma_m)}<0.01
$$
则要求
$$
\Gamma_m<0.5
$$
$$
\left(f(0.48)=0.0097404,f(0.5)=0.0105822\right)
$$

\end{document}
